\section{Grammatical evolution}
Grammatical evolution is a evolutionary computation technique used to evolve computer programs which have a high fitness in regards to a fitness function, or in other words, which do some certain task well. The phenotypes of the individuals are executable computer programs. The genotypes of the individuals are arrays of 8-bit numbers, called codons. The genotype-phenotype mapping is done using a context-free grammar which generates the language of desired solutions.

The process of genotype-phenotype mapping works as follows: we start at the first codon in individual's genotpye, and we traverse the parsing tree using depth-first search, until we find a nonterminal symbol. Once we find a nonterminal symbol, we check all productions of our specified grammar in which this nonterminal symbol is the left side (head). We calculate which production to apply on this nonterminal symbol using the formula 
$$Production\:=\:(Codon\:integer\:value)$$
$$MOD$$ 
$$(Number\:of\:productions\:for\:the\:current\:nonterminal\:symbol)$$ 
Once we apply the chosen production, we move on to the next codon. If we have reached the end of codon array, we move to the first codon, and this process is called wrapping. In practice, the maximum number of wrappings is specified, in order to avoid infinite recursions, and if the process exceeds this maximum number of wrappings, the mapping fails and the individual's fitness is set to zero.

This process of traversing the parsing tree and applying grammar productions continues until there are no more nonterminal symbols in the parsing tree, and at that point the mapping is complete \citep{neill2003grammaticalevolution}.

Grammatical evolution has some interesting unique properties, the wrapping operator and code degeneracy \citep{neill2003grammaticalevolution}. 

During the genotype-phenotype mapping process, an individual could run out of codons. In that situation, the wrapping operator is applied, which means that the mapping continues from the first codon. This technique draws inspiration from a phenomen exhibited by bacteria, viruses and mitochondria which allows them to reuse the same genetic material for the expression of different genes \citep{neill2003grammaticalevolution}.

Code degeneracy refers to the fact than the mapping process used in grammatical evolution is many-to-one. Many different codon configurations in genotype can map into the same phenotype program. For example, during the mapping process, if the current nonterminal symbol has two different productions as specified by the grammar, then the first production would be chosen if the current codon value is even, and the second production would be chosen if the current codon value is odd. This is because $0\:MOD\:2\:=2\:MOD\:2\:=4\:MOD\:2\:=6\:MOD\:2\:=\:...\:=254\:MOD\:2\:=0$, and $1\:MOD\:2\:=3\:MOD\:2\:=5\:MOD\:2\:=7\:MOD\:2\:=\:...\:=255\:MOD\:2\:=1$. The values are shown up to 254 and 255 because these are the largest even and odd numbers, respectively, that fit into the 8 bits of one codon. This property of the genotype-phenotype mapping which enables it to have many different genotypes which all map to the same phenotpye cultivates the genetic diversity of the population \citep{neill2003grammaticalevolution}.

Grammatical evolution has some drawbacks, namely low locality and high redundancy. Low locality refers to the property of the genotype-phenotype mapping that small changes in genotype can cause drastic changes in phenotype, or even completely different phenotypes. High redundancy refers to the fact that the genotype-phenotype mapping is many-to-one, which means that many different genotypes can map to a single phenotype. Because of these two properties, the evolutionary search process in grammatical evolution can sometimes behave like random search \citep{megane2022coevolutionary}. To solve these problems, some extensions of the standard grammatical evolution have been proposed, like the structured grammatical evolution \citep{lourenco2018structured}.

To demonstrate the process of genotype-phenotype mapping in the grammatical evolution, we will define a context-free grammar and one individual's genotype, and then show a step by step mapping from genotype to phenotype. Instead of defining a grammar using the $(V, T, P, S)$ 4-tuple, we will define it using the BNF notation. Terminal symbols are written as lowercase symbols, nonterminal symbols are written as uppercase symbols surrounded by symbols '<' and '>', and the first nonterminal symbol is the start symbol. Production heads are on the left side of the '::=' string, and on the right side of this string are bodies of productions which share their left side, separated by the symbol '|'. We will define our grammar as:

\noindent
$ {<}S{>}\:::=\:a\:{<}A{>}\:b\:{<}S{>}\:|\:{<}C{>}\:d\:{<}A{>}\:$\\
$ {<}A{>}\:::=\:c\:{<}B{>}\:{<}A{>}\:c\:|\:a\:b\:{<}C{>}\:|\:d $\\
$ {<}B{>}\:::=\:{<}S{>}\:a\:{<}S{>}\:|\:{<}C{>}\:d\:{<}A{>}\:|\:b\:d $\\
$ {<}C{>}\:::=\:c\:{<}C{>}\:|\:{<}C{>}\:d\:{<}C{>}\:|\:a $\\

We will also define one individual with genotype: 
$$ [\:176,\:49,\:168,\:253,\:8,\:65,\:127,\:26,\:130,\:100\:] $$

All the intermediate strings will be displayed like linear strings, which can be constructed by traversing the parsing tree in depth.

The mapping process starts with the string ${<}S{>}$, and the index $i = 0$ in the genotype array. Since $i = 0$, the codon value we will use at this step is $176$. The first nonterminal symbol in the string is ${<}S{>}$, and it has $2$ productions defined in the grammar. So we calculate $176\:\:MOD\:\:2\:=\:0$, and apply the production at index $0$, which is ${<}S{>}\:::=\:a\:{<}A{>}\:b\:{<}S{>}$.

Our current string is $a\:{<}A{>}\:b\:{<}S{>}$, the index value is $i = 1$, and the codon value is $49$. The leftmost nonterminal symbol is ${<}A{>}$, which has $3$ productions defined. So we calculate $49\:\:MOD\:\:3\:=\:1$, and apply the production at index $1$, which is ${<}A{>}\:::=\:a\:b\:{<}C{>}$

Our current string is $a\:a\:b\:{<}C{>}\:b\:{<}S{>}$, the index value is $i = 2$, and the codon value is $168$. The leftmost nonterminal symbol is ${<}C{>}$, which has $3$ productions defined. So we calculate $168\:\:MOD\:\:3\:=\:0$, and apply the production at index $0$, which is ${<}C{>}\:::=\:c\:{<}C{>}$.

Our current string is $a\:a\:b\:c\:{<}C{>}\:b\:{<}S{>}$, the index value is $i = 3$, and the codon value is $253$. The leftmost nonterminal symbol is ${<}C{>}$, which has $3$ productions defined. So we calculate $253\:\:MOD\:\:3\:=\:1$, and apply the production at index $1$, which is ${<}C{>}\:::=\:{<}C{>}\:d\:{<}C{>}$.

Our current string is $a\:a\:b\:c\:{<}C{>}\:d\:{<}C{>}\:b\:{<}S{>}$, the index value is $i = 4$, and the codon value is $8$. The leftmost nonterminal symbol is ${<}C{>}$, which has $3$ productions defined. So we calculate $8\:\:MOD\:\:3\:=\:2$, and apply the production at index $2$, which is ${<}C{>}\:::=\:a$.

Our current string is $a\:a\:b\:c\:a\:d\:{<}C{>}\:b\:{<}S{>}$, the index value is $i = 5$, and the codon value is $65$. The leftmost nonterminal symbol is ${<}C{>}$, which has $3$ productions defined. So we calculate $65\:\:MOD\:\:3\:=\:2$, and apply the production at index $2$, which is ${<}C{>}\:::=\:a$.

Our current string is $a\:a\:b\:c\:a\:d\:a\:b\:{<}S{>}$, the index value is $i = 6$, and the codon value is $127$. The leftmost nonterminal symbol is ${<}S{>}$, which has $2$ productions defined. So we calculate $127\:\:MOD\:\:2\:=\:1$, and apply the production at index $1$, which is ${<}S{>}\:::=\:{<}C{>}\:d\:{<}A{>}$.

Our current string is $a\:a\:b\:c\:a\:d\:a\:b\:{<}C{>}\:d\:{<}A{>}$, the index value is $i = 7$, and the codon value is $26$. The leftmost nonterminal symbol is ${<}C{>}$, which has $3$ productions defined. So we calculate $26\:\:MOD\:\:3\:=\:2$, and apply the production at index $2$, which is ${<}C{>}\:::=\:a$.

Our current string is $a\:a\:b\:c\:a\:d\:a\:b\:a\:d\:{<}A{>}$, the index value is $i = 8$, and the codon value is $130$. The leftmost nonterminal symbol is ${<}A{>}$, which has $3$ productions defined. So we calculate $130\:\:MOD\:\:3\:=\:1$, and apply the production at index $1$, which is ${<}A{>}\:::=\:a\:b\:{<}C{>}$.

Our current string is $a\:a\:b\:c\:a\:d\:a\:b\:a\:d\:a\:b\:{<}C{>}$, the index value is $i = 9$, and the codon value is $100$. The leftmost nonterminal symbol is ${<}C{>}$, which has $3$ productions defined. So we calculate $100\:\:MOD\:\:3\:=\:1$, and apply the production at index $1$, which is ${<}C{>}\:::=\:c\:{<}C{>}$.

Our current string is $a\:a\:b\:c\:a\:d\:a\:b\:a\:d\:a\:b\:c\:{<}C{>}$, the index value is $i = 10$. Since our codon array has only 10 elements, index $i = 10$ is out of bounds, but the mapping process is incomplete, so we apply the wrapping operator and set $i = 0$. The codon value is $176$. The leftmost nonterminal symbol is ${<}C{>}$, which has $3$ productions defined. So we calculate $176\:\:MOD\:\:3\:=\:2$, and apply the production at index $2$, which is ${<}C{>}\:::=\:a$.

Our current string is $a\:a\:b\:c\:a\:d\:a\:b\:a\:d\:a\:b\:c\:a$. Since there are no more nonterminal symbols in this string, the mapping process is finished, and this string is the final result of the mapping.